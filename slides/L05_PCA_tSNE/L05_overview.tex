\documentclass[8pt,aspectratio=169]{beamer}

% Theme
\usetheme{Madrid}
\usecolortheme{default}

% Packages
\usepackage[utf8]{inputenc}
\usepackage[T1]{fontenc}
\usepackage{amsmath,amssymb}
\usepackage{graphicx}
\usepackage{booktabs}
\usepackage{tikz}
\usepackage{hyperref}

% Custom colors (ML palette)
\definecolor{MLPurple}{RGB}{51,51,178}
\definecolor{MLBlue}{RGB}{0,102,204}
\definecolor{MLOrange}{RGB}{255,127,14}
\definecolor{MLGreen}{RGB}{44,160,44}
\definecolor{MLRed}{RGB}{214,39,40}

% Apply colors
\setbeamercolor{structure}{fg=MLPurple}
\setbeamercolor{title}{fg=MLPurple}
\setbeamercolor{frametitle}{fg=MLPurple}

% Footer customization
\setbeamertemplate{footline}{
  \leavevmode%
  \hbox{%
    \begin{beamercolorbox}[wd=.333333\paperwidth,ht=2.25ex,dp=1ex,center]{author in head/foot}%
      \usebeamerfont{author in head/foot}Methods and Algorithms
    \end{beamercolorbox}%
    \begin{beamercolorbox}[wd=.333333\paperwidth,ht=2.25ex,dp=1ex,center]{title in head/foot}%
      \usebeamerfont{title in head/foot}MSc Data Science
    \end{beamercolorbox}%
    \begin{beamercolorbox}[wd=.333333\paperwidth,ht=2.25ex,dp=1ex,right]{date in head/foot}%
      \usebeamerfont{date in head/foot}\insertframenumber{} / \inserttotalframenumber\hspace*{2ex}
    \end{beamercolorbox}}%
  \vskip0pt%
}

% Remove navigation symbols
\setbeamertemplate{navigation symbols}{}

% Custom commands
\newcommand{\bottomnote}[1]{\vfill\footnotesize\textit{#1}}
\newcommand{\highlight}[1]{\textcolor{MLOrange}{\textbf{#1}}}
\newcommand{\mathbold}[1]{\boldsymbol{#1}}

\title[L05: PCA \& t-SNE]{L05: PCA \& t-SNE}
\subtitle{Dimensionality Reduction for Visualization and Preprocessing}
\author{Methods and Algorithms}
\date{Spring 2026}

\begin{document}

\begin{frame}
\titlepage
\end{frame}

\begin{frame}{Outline}
  \tableofcontents
\end{frame}

\begin{frame}[t]{Learning Objectives}
\textbf{By the end of this lecture, you will be able to:}
\begin{enumerate}
\item Apply PCA for dimensionality reduction and feature extraction
\item Interpret variance explained and choose number of components
\item Use t-SNE for visualization of high-dimensional data
\item Compare linear (PCA) vs non-linear (t-SNE) methods
\end{enumerate}
\vspace{1em}
\textbf{Finance Application:} Portfolio risk decomposition, asset clustering
\bottomnote{From many features to meaningful low-dimensional representations}
\end{frame}

\section{Problem}

\begin{frame}[t]{The Business Problem}
\textbf{Curse of Dimensionality}
\begin{itemize}
\item Portfolio with 100+ assets: hard to visualize relationships
\item Customer data with dozens of features: redundant information
\item High dimensions cause sparsity and computational issues
\end{itemize}
\vspace{0.5em}
\textbf{Solutions}
\begin{itemize}
\item \textbf{PCA}: Linear projection preserving maximum variance
\item \textbf{t-SNE}: Non-linear embedding preserving local structure
\end{itemize}
\bottomnote{Reduce dimensions while preserving important information}
\end{frame}

\section{Method}

\begin{frame}[t]{Scree Plot: Choosing Components}
\begin{center}
\includegraphics[width=0.65\textwidth]{01_scree_plot/chart.pdf}
\end{center}
\bottomnote{Choose k components capturing 80-95\% of variance, or at the ``elbow''}
\end{frame}

\begin{frame}[t]{Principal Components}
\vspace{-0.5em}
\begin{center}
\includegraphics[width=0.55\textwidth]{02_principal_components/chart.pdf}
\end{center}
\vspace{-0.3em}
\bottomnote{Principal components are orthogonal directions of maximum variance}
\end{frame}

\begin{frame}[t]{Reconstruction Error}
\begin{center}
\includegraphics[width=0.65\textwidth]{03_reconstruction/chart.pdf}
\end{center}
\bottomnote{More components = lower error, but diminishing returns after elbow}
\end{frame}

\section{Solution}

\begin{frame}[t]{t-SNE: Perplexity Effect}
\begin{center}
\includegraphics[width=0.55\textwidth]{04b_tsne_perplexity_30/chart.pdf}
\end{center}
\bottomnote{Perplexity controls local vs global structure preservation (try 5-50)}
\end{frame}

\begin{frame}[t]{PCA vs t-SNE: Swiss Roll}
\begin{center}
\includegraphics[width=0.55\textwidth]{05b_tsne_swiss_roll/chart.pdf}
\end{center}
\bottomnote{t-SNE unrolls non-linear manifolds that PCA cannot handle}
\end{frame}

\begin{frame}[t]{Cluster Preservation}
\vspace{-0.5em}
\begin{center}
\includegraphics[width=0.55\textwidth]{06c_tsne_cluster_projection/chart.pdf}
\end{center}
\vspace{-0.3em}
\bottomnote{t-SNE better preserves cluster structure for visualization}
\end{frame}

\section{Practice}

\begin{frame}[t]{Hands-on Exercise}
  \textbf{Open the Colab Notebook}
  \begin{itemize}
    \item Exercise 1: Apply PCA to high-dimensional finance data
    \item Exercise 2: Visualize clusters with t-SNE
    \item Exercise 3: Compare PCA vs t-SNE for different datasets
  \end{itemize}
  \vspace{1em}
  \textbf{Link:} \url{https://colab.research.google.com/} [TBD]
\end{frame}

\section{Decision Framework}

\begin{frame}[t]{Decision Framework}
\vspace{-0.5em}
\begin{center}
\includegraphics[width=0.55\textwidth]{07_decision_flowchart/chart.pdf}
\end{center}
\vspace{-0.3em}
\bottomnote{PCA for preprocessing/speed, t-SNE for visualization only}
\end{frame}

\section{Summary}

\begin{frame}[t]{References}
  \footnotesize
  \begin{itemize}
    \item Jolliffe, I.T. (2002). \textit{Principal Component Analysis}. Springer.
    \item van der Maaten, L. \& Hinton, G. (2008). \textit{Visualizing Data using t-SNE}. JMLR.
    \item James et al. (2021). \textit{Introduction to Statistical Learning}. \url{https://www.statlearning.com/}
  \end{itemize}
\end{frame}

\end{document}
