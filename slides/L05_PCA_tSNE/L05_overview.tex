\documentclass[8pt,aspectratio=169]{beamer}
\usetheme{Madrid}
\usepackage{graphicx}
\usepackage{booktabs}
\usepackage{adjustbox}
\usepackage{multicol}
\usepackage{amsmath}
\usepackage{amssymb}
\usepackage{tikz}
\usepackage{hyperref}
\usepackage{algorithm}
\usepackage{algorithmic}

% Color definitions
\definecolor{mlblue}{RGB}{0,102,204}
\definecolor{mlpurple}{RGB}{51,51,178}
\definecolor{mllavender}{RGB}{173,173,224}
\definecolor{mllavender2}{RGB}{193,193,232}
\definecolor{mllavender3}{RGB}{204,204,235}
\definecolor{mllavender4}{RGB}{214,214,239}
\definecolor{mlorange}{RGB}{255, 127, 14}
\definecolor{mlgreen}{RGB}{44, 160, 44}
\definecolor{mlred}{RGB}{214, 39, 40}
\definecolor{mlgray}{RGB}{127, 127, 127}

% Additional colors for template compatibility
\definecolor{lightgray}{RGB}{240, 240, 240}
\definecolor{midgray}{RGB}{180, 180, 180}

% Backward compatibility: uppercase color names
\colorlet{MLPurple}{mlpurple}
\colorlet{MLBlue}{mlblue}
\colorlet{MLOrange}{mlorange}
\colorlet{MLGreen}{mlgreen}
\colorlet{MLRed}{mlred}
\colorlet{MLLavender}{mllavender}
\colorlet{MLGray}{mlgray}

% Apply custom colors to Madrid theme
\setbeamercolor{palette primary}{bg=mllavender3,fg=mlpurple}
\setbeamercolor{palette secondary}{bg=mllavender2,fg=mlpurple}
\setbeamercolor{palette tertiary}{bg=mllavender,fg=white}
\setbeamercolor{palette quaternary}{bg=mlpurple,fg=white}

\setbeamercolor{structure}{fg=mlpurple}
\setbeamercolor{section in toc}{fg=mlpurple}
\setbeamercolor{subsection in toc}{fg=mlblue}
\setbeamercolor{title}{fg=mlpurple}
\setbeamercolor{frametitle}{fg=mlpurple,bg=mllavender3}
\setbeamercolor{block title}{bg=mllavender2,fg=mlpurple}
\setbeamercolor{block body}{bg=mllavender4,fg=black}

% Remove navigation symbols
\setbeamertemplate{navigation symbols}{}

% Clean itemize/enumerate
\setbeamertemplate{itemize items}[circle]
\setbeamertemplate{enumerate items}[default]

% Reduce margins for more content space
\setbeamersize{text margin left=5mm,text margin right=5mm}

% Custom course footer
\setbeamertemplate{footline}{
  \leavevmode%
  \hbox{%
    \begin{beamercolorbox}[wd=.333333\paperwidth,ht=2.25ex,dp=1ex,center]{author in head/foot}%
      \usebeamerfont{author in head/foot}Methods and Algorithms
    \end{beamercolorbox}%
    \begin{beamercolorbox}[wd=.333333\paperwidth,ht=2.25ex,dp=1ex,center]{title in head/foot}%
      \usebeamerfont{title in head/foot}MSc Data Science
    \end{beamercolorbox}%
    \begin{beamercolorbox}[wd=.333333\paperwidth,ht=2.25ex,dp=1ex,right]{date in head/foot}%
      \usebeamerfont{date in head/foot}\insertframenumber{} / \inserttotalframenumber\hspace*{2ex}
    \end{beamercolorbox}}%
  \vskip0pt%
}

% Command for bottom annotation (Madrid-style)
\newcommand{\bottomnote}[1]{%
\vfill
\vspace{-2mm}
\textcolor{mllavender2}{\rule{\textwidth}{0.4pt}}
\vspace{1mm}
\footnotesize
\textbf{#1}
}

% Custom commands for course compatibility
\newcommand{\highlight}[1]{\textcolor{mlorange}{\textbf{#1}}}
\newcommand{\mathbold}[1]{\boldsymbol{#1}}

\title[L05: PCA \& t-SNE]{L05: PCA \& t-SNE}
\subtitle{Dimensionality Reduction for Visualization and Preprocessing}
\author{Methods and Algorithms}
\date{Spring 2026}

\begin{document}

\begin{frame}
\titlepage
\end{frame}

\begin{frame}{Outline}
  \tableofcontents
\end{frame}

\begin{frame}[t]{Learning Objectives}
\textbf{By the end of this lecture, you will be able to:}
\begin{enumerate}
\item \textbf{Derive} PCA from the variance maximization principle and prove the SVD--PCA equivalence
\item \textbf{Evaluate} dimensionality reduction methods (PCA vs.\ t-SNE vs.\ UMAP) for a given dataset
\item \textbf{Analyze} the effect of hyperparameters (perplexity, learning rate) on t-SNE embeddings
\item \textbf{Critique} PCA assumptions and limitations for nonlinear financial data (e.g., yield curves)
\end{enumerate}
\vspace{0.5em}
\textbf{Finance Application:} Portfolio risk decomposition, yield curve analysis, asset clustering
\bottomnote{Bloom's Level 4--5: Analyze, Evaluate, Create}
\end{frame}

\section{Problem}

\begin{frame}[t]{The Business Problem}
\textbf{Curse of Dimensionality}
\begin{itemize}
\item Portfolio with 100+ assets: hard to visualize relationships
\item Customer data with dozens of features: redundant information
\item High dimensions cause sparsity and computational issues
\end{itemize}
\vspace{0.5em}
\textbf{Solutions}
\begin{itemize}
\item \textbf{PCA}: Linear projection preserving maximum variance
\item \textbf{t-SNE}: Non-linear embedding preserving local structure
\end{itemize}
\bottomnote{Reduce dimensions while preserving important information}
\end{frame}

\begin{frame}[t]{Key Equations}
\textbf{Covariance Matrix} (from mean-centered data $X_c$):
\[
C = \frac{1}{n-1} X_c^\top X_c
\]

\textbf{Eigendecomposition}: $C\, v_k = \lambda_k\, v_k$ \quad (principal directions \& variances)

\textbf{Explained Variance Ratio}:
\[
\text{EVR}_k = \frac{\lambda_k}{\sum_{j=1}^{p} \lambda_j}
\]

\textbf{t-SNE High-Dimensional Similarity}:
\[
p_{j|i} = \frac{\exp\!\bigl(-\|x_i - x_j\|^2 / 2\sigma_i^2\bigr)}{\sum_{k \neq i}\exp\!\bigl(-\|x_i - x_k\|^2 / 2\sigma_i^2\bigr)}
\]

\bottomnote{PCA: linear eigen-problem; t-SNE: probabilistic neighbor embedding}
\end{frame}

\section{Method}

\begin{frame}[t]{Scree Plot: Choosing Components}
\begin{center}
\includegraphics[width=0.65\textwidth]{01_scree_plot/chart.pdf}
\end{center}
\bottomnote{Choose k components capturing 80-95\% of variance, or at the ``elbow''}
\end{frame}

\begin{frame}[t]{Principal Components}
\vspace{-0.5em}
\begin{center}
\includegraphics[width=0.55\textwidth]{02_principal_components/chart.pdf}
\end{center}
\vspace{-0.3em}
\bottomnote{Principal components are orthogonal directions of maximum variance}
\end{frame}

\begin{frame}[t]{Reconstruction Error}
\begin{center}
\includegraphics[width=0.65\textwidth]{03_reconstruction/chart.pdf}
\end{center}
\bottomnote{More components = lower error, but diminishing returns after elbow}
\end{frame}

\section{Solution}

\begin{frame}[t]{t-SNE: Perplexity Effect}
\begin{center}
\includegraphics[width=0.55\textwidth]{04b_tsne_perplexity_30/chart.pdf}
\end{center}
\bottomnote{Perplexity controls local vs global structure preservation (try 5-50)}
\end{frame}

\begin{frame}[t]{PCA vs t-SNE: Swiss Roll}
\begin{center}
\includegraphics[width=0.55\textwidth]{05b_tsne_swiss_roll/chart.pdf}
\end{center}
\bottomnote{t-SNE unrolls non-linear manifolds that PCA cannot handle}
\end{frame}

\begin{frame}[t]{Cluster Preservation}
\vspace{-0.5em}
\begin{center}
\includegraphics[width=0.55\textwidth]{06c_tsne_cluster_projection/chart.pdf}
\end{center}
\vspace{-0.3em}
\bottomnote{t-SNE on MNIST digits: clear digit clusters vs.\ overlapping PCA projection}
\end{frame}

\section{Practice}

\begin{frame}[t]{Hands-on Exercise}
  \textbf{Open the Colab Notebook}
  \begin{itemize}
    \item Exercise 1: Apply PCA to high-dimensional finance data
    \item Exercise 2: Visualize clusters with t-SNE
    \item Exercise 3: Compare PCA vs t-SNE for different datasets
  \end{itemize}
  \vspace{1em}
  \textbf{Link:} \url{https://colab.research.google.com/} See course materials
\end{frame}

\section{Decision Framework}

\begin{frame}[t]{Decision Framework}
\vspace{-0.5em}
\begin{center}
\includegraphics[width=0.55\textwidth]{07_decision_flowchart/chart.pdf}
\end{center}
\vspace{-0.3em}
\bottomnote{PCA for preprocessing/speed, t-SNE for visualization only}
\end{frame}

\section{Summary}

\begin{frame}[t]{References}
  \footnotesize
  \begin{itemize}
    \item Jolliffe, I.T. (2002). \textit{Principal Component Analysis}. Springer.
    \item van der Maaten, L. \& Hinton, G. (2008). \textit{Visualizing Data using t-SNE}. JMLR.
    \item James et al. (2021). \textit{Introduction to Statistical Learning}. \url{https://www.statlearning.com/}
  \end{itemize}
\end{frame}

\end{document}
