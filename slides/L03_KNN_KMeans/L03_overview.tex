\documentclass[8pt,aspectratio=169]{beamer}

% Theme
\usetheme{Madrid}
\usecolortheme{default}

% Packages
\usepackage[utf8]{inputenc}
\usepackage[T1]{fontenc}
\usepackage{amsmath,amssymb}
\usepackage{graphicx}
\usepackage{booktabs}
\usepackage{tikz}
\usepackage{hyperref}

% Custom colors (ML palette)
\definecolor{MLPurple}{RGB}{51,51,178}
\definecolor{MLBlue}{RGB}{0,102,204}
\definecolor{MLOrange}{RGB}{255,127,14}
\definecolor{MLGreen}{RGB}{44,160,44}
\definecolor{MLRed}{RGB}{214,39,40}

% Apply colors
\setbeamercolor{structure}{fg=MLPurple}
\setbeamercolor{title}{fg=MLPurple}
\setbeamercolor{frametitle}{fg=MLPurple}

% Footer customization
\setbeamertemplate{footline}{
  \leavevmode%
  \hbox{%
    \begin{beamercolorbox}[wd=.333333\paperwidth,ht=2.25ex,dp=1ex,center]{author in head/foot}%
      \usebeamerfont{author in head/foot}Methods and Algorithms
    \end{beamercolorbox}%
    \begin{beamercolorbox}[wd=.333333\paperwidth,ht=2.25ex,dp=1ex,center]{title in head/foot}%
      \usebeamerfont{title in head/foot}MSc Data Science
    \end{beamercolorbox}%
    \begin{beamercolorbox}[wd=.333333\paperwidth,ht=2.25ex,dp=1ex,right]{date in head/foot}%
      \usebeamerfont{date in head/foot}\insertframenumber{} / \inserttotalframenumber\hspace*{2ex}
    \end{beamercolorbox}}%
  \vskip0pt%
}

% Remove navigation symbols
\setbeamertemplate{navigation symbols}{}

% Custom commands
\newcommand{\bottomnote}[1]{\vfill\footnotesize\textit{#1}}
\newcommand{\highlight}[1]{\textcolor{MLOrange}{\textbf{#1}}}
\newcommand{\mathbold}[1]{\boldsymbol{#1}}

\title[L03: KNN \& K-Means]{L03: K-Nearest Neighbors \& K-Means}
\subtitle{Classification and Clustering with Distance}
\author{Methods and Algorithms}
\date{Spring 2026}

\begin{document}

\begin{frame}
\titlepage
\end{frame}

\begin{frame}{Outline}
  \tableofcontents
\end{frame}

\section{Problem}

\begin{frame}[t]{Learning Objectives}
\textbf{By the end of this lecture, you will be able to:}
\begin{enumerate}
\item Apply KNN for classification with appropriate K selection
\item Implement K-Means clustering and evaluate cluster quality
\item Compare distance metrics and their effects on results
\item Distinguish between supervised (KNN) and unsupervised (K-Means)
\end{enumerate}
\vspace{1em}
\textbf{Finance Applications:} Customer segmentation, fraud detection
\bottomnote{From parametric models (regression) to instance-based methods}
\end{frame}

\begin{frame}[t]{The Business Problem}
\textbf{Two Distinct Problems}

\textbf{1. Classification (Supervised)}
\begin{itemize}
\item Given labeled examples: is this transaction fraudulent?
\item ``Show me similar past transactions and their outcomes''
\end{itemize}
\vspace{0.5em}
\textbf{2. Clustering (Unsupervised)}
\begin{itemize}
\item No labels: what natural customer segments exist?
\item ``Group customers by behavior for targeted marketing''
\end{itemize}
\bottomnote{KNN = classification with labels, K-Means = clustering without labels}
\end{frame}

\section{Method}

\begin{frame}[t]{KNN: Decision Boundaries}
\begin{center}
\includegraphics[width=0.65\textwidth]{01_knn_boundaries/chart.pdf}
\end{center}
\bottomnote{KNN creates non-linear, flexible decision boundaries based on local data}
\end{frame}

\begin{frame}[t]{Distance Metrics}
\begin{center}
\includegraphics[width=0.50\textwidth]{02_distance_metrics/chart.pdf}
\end{center}
\bottomnote{Choice of metric affects which points are considered ``nearest''}
\end{frame}

\begin{frame}[t]{K-Means: The Algorithm}
\begin{center}
\includegraphics[width=0.65\textwidth]{03_kmeans_iteration/chart.pdf}
\end{center}
\bottomnote{Iteratively assign points and update centroids until convergence}
\end{frame}

\begin{frame}[t]{Choosing K: Elbow Method}
\begin{center}
\includegraphics[width=0.65\textwidth]{04_elbow_method/chart.pdf}
\end{center}
\bottomnote{Look for the ``elbow'' where adding clusters gives diminishing returns}
\end{frame}

\section{Solution}

\begin{frame}[t]{Cluster Quality: Silhouette Analysis}
\begin{center}
\includegraphics[width=0.55\textwidth]{05_silhouette/chart.pdf}
\end{center}
\bottomnote{Silhouette score measures how similar points are to their own cluster}
\end{frame}

\begin{frame}[t]{K-Means Decision Regions}
\begin{center}
\includegraphics[width=0.65\textwidth]{06_voronoi/chart.pdf}
\end{center}
\bottomnote{Each region contains all points closest to one centroid}
\end{frame}

\section{Practice}

\begin{frame}[t]{Hands-on Exercise}
  \textbf{Open the Colab Notebook}
  \begin{itemize}
    \item Exercise 1: Implement KNN classifier from scratch
    \item Exercise 2: Apply K-Means to customer segmentation data
    \item Exercise 3: Compare distance metrics and k values
  \end{itemize}
  \vspace{1em}
  \textbf{Link:} \url{https://colab.research.google.com/} [TBD]
\end{frame}

\section{Decision Framework}

\begin{frame}[t]{Decision Framework}
\begin{center}
\includegraphics[width=0.65\textwidth]{07_decision_flowchart/chart.pdf}
\end{center}
\bottomnote{KNN for labeled data classification, K-Means for unlabeled clustering}
\end{frame}

\section{Summary}

\begin{frame}[t]{Key Takeaways}
  \textbf{Remember}
  \begin{itemize}
    \item KNN: supervised classification using nearest neighbors
    \item K-Means: unsupervised clustering with iterative centroids
    \item Distance metrics and K selection are critical choices
    \item Finance use cases: fraud detection, customer segmentation
  \end{itemize}
  \bottomnote{Next lecture: L04 Random Forests}
\end{frame}

\end{document}
