\documentclass[8pt,aspectratio=169]{beamer}
\usetheme{Madrid}
\usepackage{graphicx}
\usepackage{booktabs}
\usepackage{adjustbox}
\usepackage{multicol}
\usepackage{amsmath}
\usepackage{amssymb}
\usepackage{tikz}
\usepackage{hyperref}
\usepackage{algorithm}
\usepackage{algorithmic}

% Color definitions
\definecolor{mlblue}{RGB}{0,102,204}
\definecolor{mlpurple}{RGB}{51,51,178}
\definecolor{mllavender}{RGB}{173,173,224}
\definecolor{mllavender2}{RGB}{193,193,232}
\definecolor{mllavender3}{RGB}{204,204,235}
\definecolor{mllavender4}{RGB}{214,214,239}
\definecolor{mlorange}{RGB}{255, 127, 14}
\definecolor{mlgreen}{RGB}{44, 160, 44}
\definecolor{mlred}{RGB}{214, 39, 40}
\definecolor{mlgray}{RGB}{127, 127, 127}

% Additional colors for template compatibility
\definecolor{lightgray}{RGB}{240, 240, 240}
\definecolor{midgray}{RGB}{180, 180, 180}

% Backward compatibility: uppercase color names
\colorlet{MLPurple}{mlpurple}
\colorlet{MLBlue}{mlblue}
\colorlet{MLOrange}{mlorange}
\colorlet{MLGreen}{mlgreen}
\colorlet{MLRed}{mlred}
\colorlet{MLLavender}{mllavender}
\colorlet{MLGray}{mlgray}

% Apply custom colors to Madrid theme
\setbeamercolor{palette primary}{bg=mllavender3,fg=mlpurple}
\setbeamercolor{palette secondary}{bg=mllavender2,fg=mlpurple}
\setbeamercolor{palette tertiary}{bg=mllavender,fg=white}
\setbeamercolor{palette quaternary}{bg=mlpurple,fg=white}

\setbeamercolor{structure}{fg=mlpurple}
\setbeamercolor{section in toc}{fg=mlpurple}
\setbeamercolor{subsection in toc}{fg=mlblue}
\setbeamercolor{title}{fg=mlpurple}
\setbeamercolor{frametitle}{fg=mlpurple,bg=mllavender3}
\setbeamercolor{block title}{bg=mllavender2,fg=mlpurple}
\setbeamercolor{block body}{bg=mllavender4,fg=black}

% Remove navigation symbols
\setbeamertemplate{navigation symbols}{}

% Clean itemize/enumerate
\setbeamertemplate{itemize items}[circle]
\setbeamertemplate{enumerate items}[default]

% Reduce margins for more content space
\setbeamersize{text margin left=5mm,text margin right=5mm}

% Custom course footer
\setbeamertemplate{footline}{
  \leavevmode%
  \hbox{%
    \begin{beamercolorbox}[wd=.333333\paperwidth,ht=2.25ex,dp=1ex,center]{author in head/foot}%
      \usebeamerfont{author in head/foot}Methods and Algorithms
    \end{beamercolorbox}%
    \begin{beamercolorbox}[wd=.333333\paperwidth,ht=2.25ex,dp=1ex,center]{title in head/foot}%
      \usebeamerfont{title in head/foot}MSc Data Science
    \end{beamercolorbox}%
    \begin{beamercolorbox}[wd=.333333\paperwidth,ht=2.25ex,dp=1ex,right]{date in head/foot}%
      \usebeamerfont{date in head/foot}\insertframenumber{} / \inserttotalframenumber\hspace*{2ex}
    \end{beamercolorbox}}%
  \vskip0pt%
}

% Command for bottom annotation (Madrid-style)
\newcommand{\bottomnote}[1]{%
\vfill
\vspace{-2mm}
\textcolor{mllavender2}{\rule{\textwidth}{0.4pt}}
\vspace{1mm}
\footnotesize
\textbf{#1}
}

% Compact list for dense slides
\newenvironment{compactlist}{%
  \begin{itemize}%
    \setlength{\itemsep}{2pt}%
    \setlength{\parskip}{0pt}%
    \setlength{\parsep}{0pt}%
}{%
  \end{itemize}%
}

% Custom commands for course compatibility
\newcommand{\highlight}[1]{\textcolor{mlorange}{\textbf{#1}}}
\newcommand{\mathbold}[1]{\boldsymbol{#1}}

\title[L03: KNN \& K-Means]{L03: K-Nearest Neighbors \& K-Means}
\subtitle{Overview}
\author{Methods and Algorithms}
\institute{MSc Data Science}
\date{Spring 2026}

\begin{document}

% ============================================================================
% INTRO ZONE (slides 1-9) — Zero Greek letters, Zero Formulas
% ============================================================================

% SLIDE 1: Title Page
\begin{frame}
\titlepage
\end{frame}

% SLIDE 2: Outline
\begin{frame}{Outline}
\tableofcontents
\end{frame}

% SLIDE 3: Can a Machine Learn from Its Neighbors? (L8 mixed media)
\begin{frame}[t]{Can a Machine Learn from Its Neighbors?}
\begin{columns}[T]
\column{0.48\textwidth}
\includegraphics[width=0.75\textwidth]{images/1838_machine_learning.png}

\column{0.48\textwidth}
\begin{itemize}
  \item Banks have millions of past transactions --- some fraudulent, most legitimate
  \item How do we classify a brand-new transaction as fraud or not?
  \item How do we find natural customer groups when nobody has labeled them?
\end{itemize}
\end{columns}
\bottomnote{XKCD \#1838 by Randall Munroe (CC BY-NC 2.5) --- Today: learning from neighbors and discovering groups}
\end{frame}

% SLIDE 4: What Is Classification? (L7 full width)
\begin{frame}[t]{What Is Classification?}
Classification is like \textbf{sorting email into spam or not-spam}: we learn from labeled examples, then predict labels for new ones.

\vspace{3mm}
\textbf{Everyday examples:}
\begin{itemize}
  \item \textbf{Spam detection:} past emails labeled spam/not-spam train a filter for new emails
  \item \textbf{Medical diagnosis:} patient records with known conditions help diagnose new patients
  \item \textbf{Fraud flagging:} historical transactions labeled fraud/legit guide future alerts
\end{itemize}

\bottomnote{Classification = supervised learning: we learn from labeled examples}
\end{frame}

% SLIDE 5: What Is Clustering? (L7 full width)
\begin{frame}[t]{What Is Clustering?}
Clustering is like \textbf{organizing a messy bookshelf by topic} --- there are no labels, you just find natural groups based on similarity.

\vspace{3mm}
\textbf{Everyday examples:}
\begin{itemize}
  \item \textbf{Customer segments:} group shoppers by purchasing behavior without predefined categories
  \item \textbf{News article grouping:} bundle related stories together automatically
  \item \textbf{Song playlists:} streaming services group songs by style and mood
\end{itemize}

\bottomnote{Clustering = unsupervised learning: discover structure without labels}
\end{frame}

% SLIDE 6: Classification vs Clustering Side by Side (L10 comparison)
\begin{frame}[t]{Classification vs Clustering Side by Side}
\begin{columns}[T]
\column{0.46\textwidth}
\textbf{\textcolor{mlblue}{Classification}}
\begin{itemize}
  \item Has labeled training data
  \item Goal: \textbf{predict} a label for new data
  \item Supervised learning --- a teacher provides answers
\end{itemize}

\column{0.46\textwidth}
\textbf{\textcolor{mlorange}{Clustering}}
\begin{itemize}
  \item No labels at all
  \item Goal: \textbf{discover} natural groups
  \item Unsupervised learning --- no teacher, just patterns
\end{itemize}
\end{columns}

\bottomnote{KNN solves classification; K-Means solves clustering --- same letter K, different meanings}
\end{frame}

% SLIDE 7: Why Do Similar Things Behave Similarly? (L7 full width)
\begin{frame}[t]{Why Do Similar Things Behave Similarly?}
This is an intuition we use every day:

\vspace{3mm}
\begin{itemize}
  \item \textbf{Medicine:} patients with similar symptoms tend to have similar diagnoses
  \item \textbf{Real estate:} houses in similar neighborhoods tend to have similar prices
  \item \textbf{Banking:} borrowers with similar financial profiles tend to have similar default rates
\end{itemize}

\vspace{3mm}
The core idea: \textbf{nearness in data space implies similarity in outcome.}

\bottomnote{KNN turns this everyday intuition into a precise algorithm}
\end{frame}

% SLIDE 8: Why Do Banks Need Customer Segments? (L7 full width)
\begin{frame}[t]{Why Do Banks Need Customer Segments?}
\begin{itemize}
  \item \textbf{Targeted products:} premium cards for high-value customers, starter accounts for new ones
  \item \textbf{Risk management:} group loans by risk profile to set appropriate interest rates
  \item \textbf{Marketing:} design retention campaigns for at-risk customers before they leave
\end{itemize}

\vspace{3mm}
\textbf{Real-world impact:} A major UK bank identified 6 customer segments, increasing product cross-sell by 15\%.

\bottomnote{Segmentation transforms raw data into actionable business strategy}
\end{frame}

% SLIDE 9: What Will You Learn Today? (L7 full width)
\begin{frame}[t]{What Will You Learn Today?}
By the end of this lecture, you will be able to answer:

\vspace{3mm}
\begin{enumerate}
  \item How does KNN use neighbors to classify new data?
  \item How does K-Means find clusters step by step?
  \item How do you choose the right value of K for either method?
  \item When should you use KNN vs K-Means?
\end{enumerate}

\vspace{4mm}
\textbf{Road Map:} Problem $\rightarrow$ Method $\rightarrow$ Solution $\rightarrow$ Practice

\bottomnote{By end of lecture: classify with KNN, cluster with K-Means, choose K}
\end{frame}

% ============================================================================
% CORE ZONE
% ============================================================================

\section{Problem}

% SLIDE 10: The Two Problems We Solve Today (L10 comparison)
\begin{frame}[t]{The Two Problems We Solve Today}
\begin{columns}[T]
\column{0.46\textwidth}
\textbf{\textcolor{mlblue}{Fraud Detection}}
\begin{itemize}
  \item Labeled transactions (fraud / legitimate)
  \item Goal: predict whether a new transaction is fraud
  \item This is \textbf{classification}
\end{itemize}

\column{0.46\textwidth}
\textbf{\textcolor{mlorange}{Customer Segmentation}}
\begin{itemize}
  \item No labels --- just raw customer data
  \item Goal: discover natural groups of customers
  \item This is \textbf{clustering}
\end{itemize}
\end{columns}

\vspace{4mm}
\textbf{\textcolor{mlred}{Important:} K in KNN = number of neighbors; K in K-Means = number of clusters --- completely different!}

\bottomnote{Same letter K, fundamentally different meanings --- watch for this!}
\end{frame}

% SLIDE 11: A Concrete Example: 5 Customers (L9 definition-example)
\begin{frame}[t]{A Concrete Example: 5 Customers}
\begin{columns}[T]
\column{0.46\textwidth}
\textbf{Training Data}

\vspace{2mm}
\begin{tabular}{ccl}
\toprule
Age & Income & Label \\
\midrule
25 & 30k & Legit \\
30 & 50k & Legit \\
45 & 80k & Fraud \\
50 & 90k & Fraud \\
35 & 40k & Legit \\
\bottomrule
\end{tabular}

\column{0.46\textwidth}
\textbf{The Question}

\vspace{2mm}
A new Customer \#6 arrives: Age = 42, Income = 75k.

\vspace{3mm}
\textbf{Which group does Customer \#6 belong to?}

\vspace{3mm}
Imagine plotting these 5 points on a scatter plot --- Customer \#6 sits near the fraud cases. KNN formalizes this intuition.
\end{columns}

\bottomnote{This small example is the starting point --- KNN scales this to millions}
\end{frame}

% ============================================================================

\section{Method}

% --- KNN (5 slides) ---

% SLIDE 12: How Does KNN Work? Three Simple Steps (L11 step-by-step)
\begin{frame}[t]{How Does KNN Work? Three Simple Steps}
\begin{enumerate}
  \item \textbf{Measure distance} to every past example in the training data
  \item \textbf{Find the $K{=}3$ closest neighbors} --- the 3 most similar past cases
  \item \textbf{Take a majority vote} among those neighbors
\end{enumerate}

\vspace{4mm}
\textbf{Worked example:} Customer \#6 has 3 nearest neighbors: 2 fraud, 1 legit $\rightarrow$ \textbf{predict fraud} (majority wins).

\vspace{3mm}
KNN is called a ``lazy learner'' (delays all computation until prediction time) --- it stores all training data and only computes distances when asked.

\bottomnote{KNN is ``lazy'': it stores all data and only computes at prediction time}
\end{frame}

% SLIDE 13: KNN Step by Step: A Worked Example (L22 chart + explanations)
\begin{frame}[t]{KNN Step by Step: A Worked Example}
\begin{columns}[T]
\column{0.55\textwidth}
\includegraphics[width=\textwidth]{09_knn_step_by_step/chart.pdf}

\column{0.42\textwidth}
\begin{itemize}
  \item We measure distance from the new point (star) to all training points
  \item The 3 closest neighbors are 2 blue, 1 orange $\rightarrow$ majority vote = blue
  \item If we used $K{=}5$ neighbors, the vote might change --- K matters!
\end{itemize}
\end{columns}

\bottomnote{The choice of K directly controls how many neighbors vote}
\end{frame}

% SLIDE 14: What Happens When K Changes? (L22 chart + explanations)
\begin{frame}[t]{What Happens When K Changes?}
\begin{columns}[T]
\column{0.55\textwidth}
\includegraphics[width=\textwidth]{01_knn_boundaries/chart.pdf}

\column{0.42\textwidth}
\begin{itemize}
  \item $K{=}1$: jagged boundary follows every point --- may memorize noise (overfitting)
  \item $K{=}15$: smooth boundary --- may miss important patterns (underfitting)
  \item Sweet spot: use cross-validation (testing on held-out data) to find the best K
\end{itemize}
\end{columns}

\bottomnote{Small K = flexible but noisy; large K = smooth but may miss detail}
\end{frame}

% SLIDE 15: How Do We Measure Distance? (L9 definition-example)
\begin{frame}[t]{How Do We Measure Distance?}
\begin{columns}[T]
\column{0.46\textwidth}
\textbf{Euclidean Distance}

\vspace{2mm}
The straight-line distance between two points:
$$d = \sqrt{\sum_{i=1}^{p} (a_i - b_i)^2}$$

\vspace{1mm}
where $a$ and $b$ are two data points with $p$ features.

\column{0.46\textwidth}
\textbf{Worked Example}

\vspace{2mm}
Customer A: age = 30, income = 50k \\
Customer B: age = 25, income = 55k

\vspace{2mm}
$d = \sqrt{(30{-}25)^2 + (50{-}55)^2}$

$d = \sqrt{25 + 25} = \sqrt{50} \approx 7.07$
\end{columns}

\vspace{3mm}
\textbf{\textcolor{mlred}{Warning:}} Always standardize features (rescale to similar ranges) first --- otherwise income (thousands) dominates age (decades)!

\bottomnote{Always standardize features first --- otherwise income (thousands) dominates age (decades)}
\end{frame}

% SLIDE 16: What Do Different Distance Measures Look Like? (L22 chart + explanations)
\begin{frame}[t]{What Do Different Distance Measures Look Like?}
\begin{columns}[T]
\column{0.55\textwidth}
\includegraphics[width=\textwidth]{02_distance_metrics/chart.pdf}

\column{0.42\textwidth}
\begin{itemize}
  \item Euclidean draws circular neighborhoods --- straight-line distance
  \item Manhattan draws diamond neighborhoods --- block-by-block distance
  \item The choice of distance metric (way of measuring) changes which points count as ``nearest''
\end{itemize}
\end{columns}

\bottomnote{Different metrics = different neighborhoods = different predictions}
\end{frame}

% --- K-Means (5 slides) ---

% SLIDE 17: How Does K-Means Find Clusters? (L11 step-by-step)
\begin{frame}[t]{How Does K-Means Find Clusters?}
Three steps, repeated until nothing changes:

\vspace{3mm}
\begin{enumerate}
  \item \textbf{Pick $K{=}3$ starting points} called centroids (cluster centers)
  \item \textbf{Assign each data point} to its nearest centroid
  \item \textbf{Move each centroid} to the center (average position) of its group
\end{enumerate}

\vspace{3mm}
Repeat steps 2--3 until centroids stop moving.

\vspace{3mm}
\textbf{Analogy:} Like rearranging seats at a party until everyone is closest to their table center.

\bottomnote{K-Means is like rearranging seats at a party until everyone is closest to their table center}
\end{frame}

% SLIDE 18: Watching K-Means Iterate (L22 chart + explanations)
\begin{frame}[t]{Watching K-Means Iterate}
\begin{columns}[T]
\column{0.55\textwidth}
\includegraphics[width=\textwidth]{03_kmeans_iteration/chart.pdf}

\column{0.42\textwidth}
\begin{itemize}
  \item Stars show centroids (cluster centers) moving toward their groups
  \item Colors show which centroid each point is assigned to
  \item After a few rounds, centroids stop moving --- we have found our clusters
\end{itemize}
\end{columns}

\bottomnote{Convergence (stopping) is guaranteed --- the algorithm always finishes}
\end{frame}

% SLIDE 19: How Do We Choose K Clusters? The Elbow Method (L22 chart + explanations)
\begin{frame}[t]{How Do We Choose K Clusters? The Elbow Method}
\begin{columns}[T]
\column{0.55\textwidth}
\includegraphics[width=\textwidth]{04_elbow_method/chart.pdf}

\column{0.42\textwidth}
\begin{itemize}
  \item WCSS (total distance from points to their centroids) always decreases with more clusters
  \item The ``elbow'' is where adding more clusters stops helping much
  \item Here $K{=}3$ clusters looks like the best choice --- diminishing returns after that
\end{itemize}
\end{columns}

\bottomnote{The elbow is sometimes hard to see --- that is why we also use silhouette analysis}
\end{frame}

% SLIDE 20: Why Does Starting Position Matter? (L7 full width)
\begin{frame}[t]{Why Does Starting Position Matter?}
\textbf{K-Means++ in plain English:} Instead of picking random starting centroids, spread them out --- pick each new centroid far from existing ones.

\vspace{3mm}
\textbf{Why it matters:} Random starts can lead to bad results because centroids may clump together in one region of the data.

\vspace{3mm}
\textbf{Analogy:} Imagine choosing 3 meeting points in a city --- you would spread them out, not put all 3 on the same street.

\vspace{3mm}
K-Means++ is the default in Python's scikit-learn library --- you get it automatically.

\bottomnote{K-Means++ is the default in scikit-learn --- always use it}
\end{frame}

% SLIDE 21: K-Means Has Limitations --- Know Them! (L18 pros/cons)
\begin{frame}[t]{K-Means Has Limitations --- Know Them!}
\textbf{\textcolor{mlgreen}{Strengths}}
\begin{itemize}
  \item[\textcolor{mlgreen}{[+]}] Fast, simple, and easy to interpret
  \item[\textcolor{mlgreen}{[+]}] Works well on round (spherical) clusters of similar size
  \item[\textcolor{mlgreen}{[+]}] Scales to large datasets
\end{itemize}

\vspace{3mm}
\textbf{\textcolor{mlred}{Limitations}}
\begin{itemize}
  \item[\textcolor{mlred}{[-]}] Assumes round clusters of similar size --- struggles with elongated shapes
  \item[\textcolor{mlred}{[-]}] Sensitive to outliers (extreme values) --- one extreme point shifts the centroid
  \item[\textcolor{mlred}{[-]}] Must specify K clusters in advance (unlike DBSCAN, a density-based method)
\end{itemize}

\bottomnote{If clusters look elongated or have very different sizes, K-Means may not be the right tool}
\end{frame}

% ============================================================================

\section{Solution}

% SLIDE 22: How Do KNN and K-Means Compare? (L3 comparison table)
\begin{frame}[t]{How Do KNN and K-Means Compare?}
\begin{center}
\begin{tabular}{lll}
\toprule
\textbf{Property} & \textbf{KNN} & \textbf{K-Means} \\
\midrule
Task & Classification & Clustering \\
Learning & Supervised (has labels) & Unsupervised (no labels) \\
K means & Number of neighbors & Number of clusters \\
Training & None --- lazy learner & Iterative centroid updates \\
Output & Class label & Cluster ID \\
\bottomrule
\end{tabular}
\end{center}

\bottomnote{Despite sharing K, these solve fundamentally different problems}
\end{frame}

% SLIDE 23: Finance: Customer Segmentation with K-Means (L22 chart + explanations)
\begin{frame}[t]{Finance: Customer Segmentation with K-Means}
\begin{columns}[T]
\column{0.55\textwidth}
\includegraphics[width=\textwidth]{11_rfm_scatter/chart.pdf}

\column{0.42\textwidth}
\begin{itemize}
  \item RFM = Recency (days since last purchase), Frequency (number of purchases), Monetary (total spend)
  \item K-Means groups customers: Champions (high R,F,M) get loyalty rewards; At-Risk (low R) get retention campaigns
  \item \textbf{Example:} Customer A: R=5, F=20, M=\$5000 $\rightarrow$ Champion. Customer B: R=180, F=2, M=\$50 $\rightarrow$ At-risk
\end{itemize}
\end{columns}

\bottomnote{Each segment gets tailored products and communication --- segments drive strategy}
\end{frame}

% SLIDE 24: Finance: Can KNN Detect Fraud? (L7 full width)
\begin{frame}[t]{Finance: Can KNN Detect Fraud?}
\textbf{The class imbalance (unequal groups) problem:} Fraud is less than 1\% of transactions.

\vspace{3mm}
If KNN predicts ``no fraud'' for everything $\rightarrow$ 99\% accuracy but catches \textbf{zero} fraud cases!

\vspace{3mm}
\textbf{Solutions:}
\begin{itemize}
  \item \textbf{Oversample} fraud cases --- duplicate rare examples to balance the training data
  \item \textbf{Weight fraud neighbors} more heavily --- give more influence to minority class votes
  \item \textbf{Measure with Precision-Recall} instead of accuracy --- reward catching actual fraud
\end{itemize}

\bottomnote{In fraud detection, missing a fraud case is far more costly than a false alarm}
\end{frame}

% SLIDE 25: Which Method Should You Choose? (L22 chart + explanations)
\begin{frame}[t]{Which Method Should You Choose?}
\begin{columns}[T]
\column{0.55\textwidth}
\includegraphics[width=\textwidth]{07_decision_flowchart/chart.pdf}

\column{0.42\textwidth}
\begin{itemize}
  \item Have labels? $\rightarrow$ supervised methods like KNN
  \item No labels, round clusters? $\rightarrow$ K-Means
  \item Weird-shaped clusters? $\rightarrow$ DBSCAN (density-based) or hierarchical clustering
\end{itemize}
\end{columns}

\bottomnote{Start simple (K-Means or KNN), add complexity only if results are poor}
\end{frame}

% ============================================================================

\section{Practice}

% SLIDE 26: Hands-on Exercise (L7 full width)
\begin{frame}[t]{Hands-on Exercise}
\begin{enumerate}
  \item \textbf{KNN Classification:} Apply KNN to classify customers --- vary K neighbors and see how predictions change
  \item \textbf{K-Means Segmentation:} Segment customers with K-Means --- interpret what each group means in business terms
  \item \textbf{Choosing K:} Compare the elbow method and silhouette analysis to choose K clusters
\end{enumerate}

\bottomnote{Exercises progress from guided implementation to open-ended analysis}
\end{frame}

% ============================================================================

\section{Summary}

% SLIDE 27: Key Takeaways (L13 summary)
\begin{frame}[t]{Key Takeaways}
\begin{columns}[T]
\column{0.46\textwidth}
\textbf{\textcolor{mlblue}{KNN}}
\begin{itemize}
  \item Non-parametric (no fixed model), lazy learner
  \item Small K = flexible but noisy; large K = smooth but may miss detail
  \item \textbf{Scale your features!}
\end{itemize}

\column{0.46\textwidth}
\textbf{\textcolor{mlorange}{K-Means}}
\begin{itemize}
  \item Iterative assign-and-update, always converges (stops)
  \item Use K-Means++ for smart starting
  \item Check K with elbow and silhouette methods
\end{itemize}
\end{columns}

\vspace{3mm}
\textbf{Both methods:} Feature scaling is critical, and K means different things in each algorithm.

\bottomnote{Get the distance right, get the result right --- both methods depend on distance}
\end{frame}

% SLIDE 28: Until Next Time... (L8 mixed media)
\begin{frame}[t]{Until Next Time...}
\begin{columns}[T]
\column{0.48\textwidth}
\includegraphics[width=0.75\textwidth]{images/2731_kmeans_clustering.png}

\column{0.48\textwidth}
\vspace{2mm}
\textit{``Even K-Means would struggle to cluster the ways students misuse K-Means.''}

\vspace{4mm}
\textbf{Next session:} L04 --- Random Forests
\end{columns}

\bottomnote{XKCD \#2731 by Randall Munroe (CC BY-NC 2.5) --- Clustering is easy; knowing when to cluster is hard}
\end{frame}

\end{document}
